\newpage
%%%%%%%%%% Section with local TOC%%%%%%%%%
\sectionwithlocaltoc{Derivations}
\setcounter{secnumdepth}{0}
%%%%%%%%%%%%%%%%%%%%%%%%%%%%%%%
\newpage




%=================================
\subsection{Raychaudhuri's Idenitity}
%=================================
Raychaudhuri's Identity will depend on a few initial definitions.


\begin{definition}{Congruence of a Vector Field}{}
A \vocab{congruence} is the set of integral curves of a non-vanishing vector field on a manifold, say \(\mc M\).
%
They may be described by a function \(F:~\mc M \times \mathbb{R} \to \mc M\).
%
We say that \(F\) is an integral curve for the vector field\(u\) passing through \(p\) at time \(t_0\) if
\begin{equation}
\begin{aligned}
    F(x; 0) &= p,
    ~~~~~~~~
    F(x; t + s) = F(F(x, t), s),
    \\
    \dv{t}F(x; t) &= u(F(x; t))
    .
\end{aligned}
\end{equation}

Of particular importance will be \vocab{timelike congruences}, which are identical up to the additional constraint that they are everywhere timelike.
\end{definition}


\begin{definition}{Orthogonal Projector to a Vector Field}{}
An \vocab{orthogonal projector} of a (non-null) vector field \(u\), sometimes called projector or projection operator, is a tensor \(h_{ab}\),
\begin{align}
    h_{ab} = g_{ab} - u_a u_b / (u^c u_c)
    ,
\end{align}
which projects to a subspace of a tangent bundle which is orthogonal to the vector field \(u^a\).

Unless otherwise indicated, we will be discussing the orthogonal projectors of \textit{unit} vector fields, which take the form
\begin{align}
    h_{ab} = g_{ab} - u_a u_b
    .
\end{align}

If \(u\) is instead null, \(u^c u_c = 0\), the orthogonal projector instead takes the form
\begin{align}
h_{ab} = 
.
\end{align}

\end{definition}

\begin{definition}{Expansion of a Vector Field}{}
The \vocab{expansion} \(\theta\) of a vector field \(u\) is defined as
    \begin{align}
        \theta = u^a_{~~;a}
        .
    \end{align}
\end{definition}



\begin{definition}{Vorticity of a Vector Field}{}
The \vocab{vorticity} \(\omega\) of a vector field \(u\) on a manifold is defined by
\begin{align}
    \omega_{ab}&= h_a^{~~c} h_b^{~~d} u_{[c;d]},
    \\
    \omega&=\frac{1}{2}\omega_{ab}\omega^{ab},
\end{align}
with \(h_{ab}\) the projector onto the subspace orthogonal to \(u^a\).
%
It characterizes the antisymmetrized derivative of \(u\) in the orthogonal subspace, or the amount to which it swirls in the orthogonal subspace.
\end{definition}


\begin{definition}{Shear of a Vector Field}{}
The \vocab{shear} \(\sigma\) of a vector field on a manifold is defined by 
\begin{align}
    \omega_{ab}
    &=
    h_a^{~~c} h_b^{~~d} u_{(c;d)}
    - \frac{1}{d-1}\theta h_{ab}
    \\
    \omega&=\frac{1}{2}\sigma_{ab}\sigma^{ab}
    ,
\end{align}
with \(h_{ab}\) the projector onto the subspace orthogonal to \(u^a\).
%
It characterizes the traceless, symmetrized derivative of \(u\) in the orthogonal subspace (the amount in which it shears against itself in the orthogonal subspace).
\end{definition}

One can similarly define concepts of normal stress for a vector field \(u\), captured by \(h_a^c u^d u_{c; d}\) and acceleration for \(u\), captured by \(u^d u^c u_{c;d}\).


With these definitions in hand, we may introduce the identity.
 \begin{theorem}{Raychaudhuri's Identity}{}
Given a timelike congruence \(u^\mu\).
%
The following identity, called \vocab{Raychaudhuri's Idenitity}, holds:
\begin{equation}
    R_{ab} u^a u^b
    =
    -\dot \theta - \frac{1}{3}\theta^2
    - 2\sigma^2 + 2\omega^2 +
    \frac{D}{Dt}\left(u^a_{~~;a}\right)
    ,
\end{equation}
where \(\frac{D}{Dt}\) indicates the derivative along the direction of \(u^\mu\): \(\frac{D}{Dt} = u^a \nabla_a\)
\end{theorem}


\begin{proof}
    Recall that the definition of the Riemann tensor gives
    \begin{align}
        u^a R_{abcd} = u_{b;cd} - u_{b;dc}
        .
    \end{align}
    
    Contracting with \(u^c\) and setting \(b = d\) by contracting with \(g^{bd}\), we have
    \begin{align}
        u^a u^c R_{ac} = u^b_{~~;cd} u^c - u^b_{~~;dc} u^c
        .
    \end{align}
    
    Using the definition of the expansion,
    \begin{align}
        \theta = u^a_{~~;a},
    \end{align}
we have
    \begin{align}
        u^a u^c R_{ac} = u^b_{~~;cd} u^c - \frac{D}{Dt}\theta
        .
    \end{align}
    
    Next, ...


\end{proof}






\begin{remark}{Naming Rights}{}
    Apparently, %https://arxiv.org/pdf/1905.01955.pdf
    Raychaudhuri's identity was first introduced by a mathematician (?) named Ehlers.
\end{remark}


% Try also https://arxiv.org/pdf/hep-th/9705083.pdf
